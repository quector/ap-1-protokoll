

\documentclass[a4paper]{scrartcl}

\usepackage[utf8]{inputenc}
\usepackage[english]{babel} % use ngerman here for german text
\usepackage[T1]{fontenc}
\usepackage{mathtools}
\usepackage{amssymb}
\usepackage{lmodern}
\usepackage[protrusion=true,expansion=true,kerning]{microtype}
\usepackage{graphicx}
\usepackage{hyperref}
\usepackage{amsmath}
\usepackage{amsfonts}
\usepackage{lmodern}
\usepackage{bm}
\usepackage{physics}
\usepackage{color}



\title{Jordan normal form}
\author{ICH BIN DER CLEMENS}
\date{\today}

\begin{document}
	\maketitle
	\tableofcontents
	\newpage
	hallo hier bin ich
	hier gehts los mit der unterhos
		\section{ICH HAB HEUTE KEIN CAPRI EIS GEGESSEN}
			hallo schön sie hier zu sehen. Sehr erfreut, ich bin der clemens und wie heißt du
			Boldface(B)
			
			\subsection{zwischentext}
			dieser zwischentext ist von elementärer bedeutung für die vollendung dieses dokuments . Wer das liest ist dumm xdddddd
				\subsubsection{über der geile shit}
				ich bin schon groß und vier, komm doch und spiel mit mir ich lad dich ein ich bin\label{eins} caiou \vspace{40pt}
				
				\begin{figure}[h]
					\centering
%					\includegraphics[width=0.7\linewidth]{C:/Users/Clemens/Downloads/wallpaper}
					\caption[ich bin eine sehr kurze unterschrift von dem bild]{ich bin eine etwas längere unterschrift unter dem bild}
					\label{fig:wallpaper}
				\end{figure}
				\pagebreak
				
				jetzt gehts los alter \textbf{ich bin fett}
				\emph{BITTESEISCHNELL}\textit{ich bin kursiv}
				\ref{eins}
				\linebreak[3]
				\\ \begin{center} \huge ${F}_{g}=\frac{G{m}_{1}h^2{m}_{2}}{{r}^{2}}$
				\linebreak[5]
				\\ {\color{red}${F}_{z}=\frac{{m}_{1}{m}_{2}{2}}{{}{m r}^{2}}$}
				\linebreak[5]
				\\$E_{potential}=\frac{{1}}{2}mgh$
				\linebreak[10]
				
				\end{center}
			
			so, weiter gehts am linken rand lol
		
		\section{endlich vorbei mit dem formel scheiß}
		\begin{tabular}{||c||c||c||c||c||c|}
	
			\hline 
			Spalte 1  & Spalte 2 & Spalte 3 & Spalte 4 & Spalte 5 & Spalte 6 \\ 
			\hline 
			Zeile 1  & Zeile 2  & Zeile 3  & Zeile 4 & Zeile 5 & Zeile 6 \\ 
			\hline 
			und  & so  & weiter  & immer  & mehr  & zeilen  \\ 
			\hline 
			ausfüllen & bis  & ich  & nicht  & mehr  & weiter \\ 
			\hline 
			weiß  & und  & mir  & keine  & worte  & mehr \\ 
			\hline 
			in  & den & kopf  & kommen  & und  & mir \\ 
			\hline
		\end{tabular} 
		
		\paragraph{yo}
		So, jetzt wo ich mich an ein paar formeln probiert habe, wird es zeit ein paar tabellen aufzustellen. Dies hier ist eine sehr komplexe und nur den besten latechsern vorbehaltene tabelle :
		\underline{holy shit ich kann meine scheiß texte jetzt sogar unterstreichen, wenn sie super wichtig sind Pog} und so sieht das aus, wenn die nächste zeile an der alten zeile mit
		 unterstrichenem text untendrann vorbeistreicht. Irgendwie hab ichs verpeilt das unterstrichene texte keinen zeilenumbruch haben ?!  1234567897654323456789876543 
		 \huge bitte keine unnötgen änderungen in dem protokoll hinzufügen sonst reporte ich dich !!!
		 \vspace{20pt}
		  \\ {\huge FIX KOMMT SPÄTER }
		
		
		
		
		\begin{enumerate}
			\item[5] der erste satz
			\item das zweite wort 
			\begin{itemize}
			\item und so weiter
			\end{itemize}
			\item ich weiß nicht was ich hier noch hin schreiben soll
		\end{enumerate}
	
	 $$1+2=3+3+h_{planck ist ein guter physiker}+4+4=M_{deine Mutter}+900 {kg}+50$$
	 \begin{equation}
	 1+2+3=M_{Gummibärchen}
	  \epsilon
	 \end{equation}
\end{document}
