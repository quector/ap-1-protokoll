\documentclass[11pt,a4paper]{article}
\usepackage{amsmath,amsfonts,amssymb}
\usepackage[ngerman]{babel}
\usepackage[T1]{fontenc}
\usepackage[utf8]{inputenc}
\usepackage{longtable}
\usepackage{ae}
\usepackage{geometry}
\geometry{verbose,a4paper,tmargin=25mm,bmargin=25mm,lmargin=20mm,rmargin=20mm,nomarginpar}
\usepackage{tikz}
\usetikzlibrary{patterns}
\usepackage{graphicx}
\usepackage{mathrsfs}
\usepackage{array}
\usepackage{paralist}
\usepackage[pdfpagelabels]{hyperref}
\usepackage{pdfpages}
\usepackage{hyperref}
\usepackage{marvosym}
\usepackage{amssymb}
\usepackage{amsmath}
\usepackage{float}


\begin{document}
\newcommand{\angstrom}{\mbox{\normalfont\AA}}
\newcommand*{\vertbar}{\rule[1ex]{0.5pt}{2.5ex}}
\newcommand*{\horzbar}{\rule[.5ex]{2.5ex}{0.5pt}}
\newcommand{\sg}{\vspace*{0.15cm}}
\newcommand{\pic}[4]{\begin{figure}[ht] \centering \includegraphics[width=#1\textwidth]{#2}\caption{#3}\label{#4}  
\end{figure}}
\newcommand{\picoc}[3]{\begin{figure}[ht] \centering \includegraphics[width=#1\textwidth]{#2}\label{#3}  
\end{figure}}

\pagenumbering{Alph}
\begin{titlepage}

\newcommand{\HRule}{\rule{\linewidth}{0.5mm}} % Defines a new command for the horizontal lines, change thickness here

\center % Center everything on the page
 
%----------------------------------------------------------------------------------------
%	HEADING SECTIONS
%----------------------------------------------------------------------------------------

\textsc{\LARGE Albert-Ludwigs-Universität Freiburg}\\[1.5cm] % Name of your university/college
\textsc{\Large Anfänger Praktikum I}\\[0.5cm] % Major heading such as course name
\textsc{\large Versuch 17}\\[0.5cm] % Minor heading such as course title

%----------------------------------------------------------------------------------------
%	TITLE SECTION
%----------------------------------------------------------------------------------------

\HRule \\[0.4cm]
{ \huge \bfseries   Physikalisches Pendel, Trägheitsmomente und Steinerscher Satz}\\[0.4cm] % Title of your document
\HRule \\[1.5cm]

%----------------------------------------------------------------------------------------
%	IMAGE
%----------------------------------------------------------------------------------------

\includegraphics[width=200pt]{logo}\\[1cm]

 
%----------------------------------------------------------------------------------------
%	AUTHOR SECTION
%----------------------------------------------------------------------------------------
\mbox{}
\vfill
\begin{minipage}{0.4\textwidth}
\begin{flushleft} \large
\emph{Versuchsteilnehmer:}\\
\underline{Andreas \textsc{Weber}},\\ \underline{Clemens \textsc{Lauby}} % Your name
\end{flushleft}
\end{minipage}
~
\begin{minipage}{0.4\textwidth}
\begin{flushright} \large
\emph{Tutor: } \\
Vorname \textsc{Nachname} % Supervisor's Name
\end{flushright}
\end{minipage}\\[1cm]

% If you don't want a supervisor, uncomment the two lines below and remove the section above
%\Large \emph{Author:}\\
%John \textsc{Smith}\\[3cm] % Your name

%----------------------------------------------------------------------------------------
%	DATE SECTION
%----------------------------------------------------------------------------------------

{\large \today}\\ % Date, change the \today to a set date if you want to be precise

%----------------------------------------------------------------------------------------
%	LOGO SECTION, if you want no logo leave this as is
%----------------------------------------------------------------------------------------

%\includegraphics[width=200pt]{logo}\\[1cm] % Include a department/university logo - this will require the graphicx package
 
%----------------------------------------------------------------------------------------

\vfill % Fill the rest of the page with whitespace
\end{titlepage}
\newpage
\pagenumbering{Alph}

\tableofcontents
\newpage
\pagenumbering{arabic}
\section{Abstract}
	















\section{Theorie}

\section{Experiment}
\subsection{Versuchsaufbau}

\subsection{Versuchsdurchführung}

\section{Auswertung}

\section{Diskussion}



\newpage
\section{Quellen}
%\bibliographystyle{apacite}

\nocite{*}
\newpage
\section{Messwerte}
\newpage
%\includepdf[pages={1-}]{abc.pdf}



\end{document}

